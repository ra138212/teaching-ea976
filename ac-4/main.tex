%%%%%%%%%%%%%%%%%%%%%%%%%%%%%%%%%%%%%%%%%%%%%%%%%%%%%%%%%%%%%%%%%%%%%%
% LaTeX Example: Project Report
%
% Source: http://www.howtotex.com
%
% Feel free to distribute this example, but please keep the referral
% to howtotex.com
% Date: March 2011 
% 
%%%%%%%%%%%%%%%%%%%%%%%%%%%%%%%%%%%%%%%%%%%%%%%%%%%%%%%%%%%%%%%%%%%%%%
% How to use writeLaTeX: 
%
% You edit the source code here on the left, and the preview on the
% right shows you the result within a few seconds.
%
% Bookmark this page and share the URL with your co-authors. They can
% edit at the same time!
%
% You can upload figures, bibliographies, custom classes and
% styles using the files menu.
%
% If you're new to LaTeX, the wikibook is a great place to start:
% http://en.wikibooks.org/wiki/LaTeX
%
%%%%%%%%%%%%%%%%%%%%%%%%%%%%%%%%%%%%%%%%%%%%%%%%%%%%%%%%%%%%%%%%%%%%%%
% Edit the title below to update the display in My Documents
%\title{Relatório Atividade Complementar FLOSS EA976}
%
%%% Preamble
\documentclass[12pt,a4paper]{article} % Use A4 paper with a 12pt font size - different paper sizes will require manual recalculation of page margins and border positions

\usepackage{marginnote} % Required for margin notes
\usepackage{wallpaper} % Required to set each page to have a background
\usepackage{lastpage} % Required to print the total number of pages
\usepackage[left=1.3cm,right=2.0cm,top=1.8cm,bottom=5.0cm,marginparwidth=3.4cm]{geometry} % Adjust page margins
\usepackage{amsmath} % Required for equation customization
\usepackage{amssymb} % Required to include mathematical symbols
\usepackage{xcolor} % Required to specify colors by name
\usepackage[utf8]{inputenc}
\usepackage{fancyhdr} % Required to customize headers
\usepackage[brazil]{babel}
\usepackage[latin1]{inputenc}
%\usepackage[T1]{fontenc}
%\usepackage{graphicx}
\usepackage{pstricks}
\usepackage{subfigure}
\usepackage{caption}  % legendas nas figuras
\captionsetup{justification=centering,labelfont=bf}
\usepackage{textcomp}
\setlength{\headheight}{80pt} % Increase the size of the header to accommodate meta-information
\pagestyle{fancy}\fancyhf{} % Use the custom header specified below
\renewcommand{\headrulewidth}{0pt} % Remove the default horizontal rule under the header

\setlength{\parindent}{0cm} % Remove paragraph indentation
\newcommand{\tab}{\hspace*{2em}} % Defines a new command for some horizontal space

\newcommand\BackgroundStructure{ % Command to specify the background of each page
\setlength{\unitlength}{1mm} % Set the unit length to millimeters

\setlength\fboxsep{0mm} % Adjusts the distance between the frameboxes and the borderlines
\setlength\fboxrule{0.5mm} % Increase the thickness of the border line
\put(10, 20pr){\fcolorbox{black}{gray!5}{\framebox(155,247){}}} % Main content box
\put(165, 20){\fcolorbox{black}{gray!10}{\framebox(37,247){}}} % Margin box
\put(10, 262){\fcolorbox{black}{white!10}{\framebox(192, 25){}}} % Header box
\put(175, 263){\includegraphics[height=23mm,keepaspectratio]{}} % Logo box - maximum height/width: 
}

%----------------------------------------------------------------------------------------
%	HEADER INFORMATION
%----------------------------------------------------------------------------------------

\fancyhead[L]{\begin{tabular}{l r | l r} % The header is a table with 4 columns
\textbf{2S/2014} & EA976 & \textbf{P\'agina:} & \thepage/\pageref{LastPage} \\ % Project name and page count
\textbf{Atividade:} & Complementar \#4 & \textbf{Data:} & 30/11/14 \\ % Job number and last updated date
\textbf{Professor:} & Christian E. Rothenberg & \textbf{Assunto:} & Caracterização  projeto \textit{open source} Debian  \\ % Version and reviewed date
\textbf{Projeto:} & Debian & \textbf{Autor:} & Daniel Meneguin Barbosa (RA: 138212) \\ % Designer and reviewer
\end{tabular}}

%----------------------------------------------------------------------------------------

\begin{document}

%\AddToShipoutPicture{\BackgroundStructure} % Set the background of each page to that specified above in the header information section

%----------------------------------------------------------------------------------------
%	DOCUMENT CONTENT
%----------------------------------------------------------------------------------------

\section{Relat\'orio t\'ecnico descritivo sobre projeto de c\'odigo livre} 

	Este relatório é uma apresentação ao projeto open source Debian e suas características relacionadas ao desenvolvimento, licença, governança e manutenção.

\subsection{Descrição do projeto}

	O projeto Debian é um software livre que consiste em um sistema operacional baseado no kernel Linux utilizado por empresas e indivíduos.


\section{Caracterização do projeto de código livre} 


\subsection{Desenvolvimento}


\begin{itemize}
\item Existe um local dedicado para o desenvolvimento?
\\*
	\\Não. A estrutura na qual os desenvolvedores são organizados permite que não seja necessária a alocação física dos mesmos para o desenvolvimento do sistema operacional. A comunicação é feita através de listas de email (https://lists.debian.org) e canal de IRC (canal \#debian em irc.debian.org). https://www.debian.org/intro/organization. A localização dos desenvolvedores pode ser verificada neste link: https://www.debian.org/devel/developers.loc
\linebreak
\item É possível extrair o atual código fonte a partir de um repositório público de código fonte?

\\*
	\\Sim. Existe um repositório mantido por membros da Debian onde o código pode ser baixado via ftp. Segue o link do repositório: https://ftp-master.debian.org/
\linebreak

\item Quão grande é o código?

\\*
\\A última versão estável do Debian (7.0) possui 419 milhões de linhas de código.
\linebreak

\item Quais são as principais linguagens de programação?
\\*
\\As linguagens mais utilizadas são C e C++. No link a seguir podem ser verificadas 15 linguagens de programação utilizadas no desenvolvimento do Debian: https://wiki.debian.org/ProgrammingLanguage
\linebreak

\item A utilização do pacote depende de algum outro software proprietário ou de código fonte aberto?

\\*
\\A utilização do Debian como sistema operacional depende de vários softwares de terceiros que por muitas vezes são proprietários ou possuem código fonte aberto. O desenvolvimento do Debian busca dar suporte a todos estes, como informa o link a seguir na sessão “É tudo realmente livre?”: https://www.debian.org/intro/about
\linebreak

\item É possível calcular o número de \textit{downloads} ou usuários de uma versão em particular?
\\*
\\Devido ao grande número de mirrors e diferentes formas de compartilhamento das versões do Debian, não é possível calcular a quantidade de downloads realizados de suas versões.
\linebreak

\end{itemize}

\subsection{Licença Software Livre}


\begin{itemize}
\item Quem são os patrocinadores que contribuem para a sustentabilidade do projeto?
\\*
\\Os principais patrocinadores do desenvolvimento do projeto são: Software in the public interest (SPI) e Umbrella organization. Mas existem muitas outras como Hewlett-Packard, trustsec , credativ GmbH, Skolelinux, Genesi, MGE UPS SYSTEMS, Simtec etc. (uma lista completa pode ser encontrada neste link: https://www.debian.org/partners/)
\linebreak


\item Quem detém os direitos autorais do código?
\\*
\\Quem detém os direitos autorais de cada pacote do Debian é o próprio autor de cada pacote.
\linebreak
\item O projeto está sob qual tipo de licença de código aberto?
\\*
\\O Debian é um Software Livre e o autor de cada pacote pode utilizar a licença que desejar, porém apenas os pacotes com licenças que respeitam o Contrato Social Debian (https://www.debian.org/social_contract) entram na distribuição principal.
\linebreak

\item Por que os responsáveis pelo projeto escolheram a licença de código aberto?
\\*
\\A licença GPL prevê que qualquer trabalho derivado deve estar sob as mesmas condições que o trabalho original. O Debian foi baseado no Kernel Linux que está sob licença GPL, portanto, uma vez que o Debian foi derivado do Linux, é obrigatório que seu código seja aberto.
\linebreak
\end{itemize}

\subsection{Governança}


\begin{itemize}
\item Existem quantos desenvolvedores alocados para o projeto?
\\*
\\O Debian é produzido por aproximadamente mil desenvolvedores ativos espalhados pelo mundo.
\linebreak
\item Quantos \textit{committers}, também conhecidos por desenvolvedores que podem realizar mudanças propostas, o projeto possui?
\\*
\\Os “comitters” do Debian são chamados de “Sponsors” e um Sponsor pode ser qualquer desenvolvedor Debian. Portanto, existem aproximadamente mil pessoas que podem realizar mudanças na versão oficial.
\linebreak
\item O que você pode dizer sobre o modelo de governança de código fonte aberto?
\\*
\\O Debian possui uma estrutura organizacional muito bem organizada. O líder do projeto é eleito uma vez por ano pela comunidade de desenvolvedores. Este possui mais poderes que os outros e podem tomar decisões levando em conta critérios técnicos e o consenso da comunidade. A comunidade pode também rever uma decisão tomada pelo líder e cada decisão é votada por todos.
\linebreak
\end{itemize}

\subsection{Manutenção}


\begin{itemize}
\item Gerenciamento de \textit{releases}: Qual o número e frequência de \textit{releases}?
\\*
\\ocorreram 12 releases do Debian. A frequência média é de uma release a cada dois anos.
\linebreak
\item Comunicação: Existe um canal de comunicação útil e ativo para a comunidade / suporte ao usuário?
\\*
\\A comunicação é feita primariamente através de e-mail (listas de discussão em https://lists.debian.org) e IRC (canal #debian em irc.debian.org).
\linebreak
\item Existe um \textit{bugtracker} (rastreamento de bugs) com uma lista de bugs corrigidos/pendentes de correção?
\\*
\\Sim. Existe uma lista que contem os bugs, estes são associados a um número e é mantido no arquivo até que seja marcado como tendo sido trabalhado. Esta lista pode ser encontrada neste link: https://www.debian.org/Bugs/
\linebreak
\item Existe um plano de metas para planos futuros? Existe evidência que o plano de metas foi utilizado no passado?
\\*
\\Cada versão do Debian possui uma página de status que se aproxima de um plano de metas. Esta página existe inclusive na próxima versão que será lançada (codinome Jessie: https://www.debian.org/releases/jessie/)
\linebreak

\item Existe consultoria comercial, treinamento ou consulta disponível para o projeto? A partir de múltiplos prestadores de serviços?         
\\*
\\Existe uma lista de consultores localizados por todo o mundo disponíveis para o desenvolvimento de funcionalidades extras para o Debian. Esta lista pode ser encontrada no seguinte link: https://www.debian.org/consultants/
\linebreak

\end{itemize}
        

\par\vspace{\baselineskip}

%----------------------------------------------------------------------------------------

\end{document}